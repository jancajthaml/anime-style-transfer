\chapter{Introduction}

\section{Motivation}

A large number of machine learning applications rely on a vast amount of data from the real world to infer useful relations. However, obtaining these data is not always a viable option. For example, you would like to teach your autonomous car to recognize approaching collision in order to avoid it, but crashing real cars in order to capture the image or depth data is not possible. Luckily, computer graphics started to become more and more realistic in recent years, and it is now possible to capture image and depth data from computer games. These data have one significant advantage -- it can simulate almost any scenario such as crashing, unusual environment, etc., as long as it is possible within the game.

Although the data captured from the modern computer games look almost realistic, it suffers from many problems to be readily usable by machine learning applications. The most significant drawback is the fact, that they look {\em too} perfect -- real-world sensors often measure data with noise or fail altogether.

Our goal in this thesis is to find such a relation between the in-game data and the real world data to be able to transform the in-game data to look as realistically as possible. Since we do not have a one-to-one mapping between these data, it is necessary to apply methods of {\em unsupervised} learning.

\section{Thesis structure}
In this first chapter, we set up motivation and reasoning for this work. The next chapter is an overview of related theoretical work. The first section of said chapter briefly summarizes recent work in the field, while the next section explores more deeply neural networks used in this thesis. The last section of this chapter describes operation of LiDAR which we are trying to simulate in chapter \ref{experiments}.

Chapter \ref{dataset} is dedicated to used datasets and is divided into two parts corresponding to depth and RGB datasets. In this chapter, we summarize key characteristics of the datasets and how they were obtained.

Chapter \ref{programs} describes all the programs written for the purpose of this thesis and shows their functionality. This chapter can also serve as a user guide for the programs.

Chapter \ref{experiments} is a showcase of performed experiments. We also describe all the drawbacks we encountered during the experiments. The chapter ends with an evaluation of results.

In the last chapter, we analyze all the results and discuss the contribution of this thesis, followed by plans for the future work and conclusion.
